
\documentclass[a4paper,10pt]{scrartcl}

\usepackage[utf8]{inputenc}
\usepackage{tikz}
\usepackage{amsmath}
\usepackage{hyperref}
\usepackage{makecell}
\usepackage{nicematrix}
\usepackage{xkeyval}
\usepackage{pdflscape}
\usepackage{longtable}
\usepackage{booktabs}% http://ctan.org/pkg/booktabs
\newcommand{\tabitem}{~~\llap{\textbullet}~~}


\usepackage{subfig}
\usepackage{graphicx}
\usepackage{pgfplots}
\usepackage{pgfplotstable}
\tikzset{>=stealth}
\usetikzlibrary{patterns}
\usetikzlibrary{pgfplots.statistics}

\setlength{\paperheight}{30cm}
\setlength{\textheight}{24.5cm}
\setlength{\paperwidth}{21cm}
\setlength{\textwidth}{16.8cm}
\setlength{\oddsidemargin}{-0.3cm}
\setlength{\evensidemargin}{-0.3cm}
\setlength{\parindent}{0cm}
\setlength{\topmargin}{-1.9cm}
\setlength{\headsep}{1.3cm}
\setlength{\footskip}{1.5cm}

\title{Functional Hazard Analysis of a Helicopter-based Highline Rescue Using a Longline}

\author{Version 2.0 by Jakob Bludau, Aaron Benkert, Lukas Irmler,\\ and Michael Seemann}

%define fha table
\makeatletter
\def\fhaSystem{Fill in cell}
\def\fhaFunction{Fill in cell}
\def\fhaCriticalityPre{Fill in cell}
\def\fhaNumber{Fill in cell}
\def\fhaCauses{Fill in cell}
\def\fhaDescription{Fill in cell}
\def\fhaMitigation{Fill in cell}
\def\fhaCriticalityPost{Fill in cell}

%\define@key{fhaSystemKey}{fhaSystemKeyValue}{%
%    \def\fhaSystem{#1}%
%}
%\define@key{fhaFunctionKey}{fhaFunctionKeyValue}{%
%    \def\fhaFunction{#1}%
%}
%\define@key{fhaCritPreKey}{fhaCritPreKeyValue}{%
%    \def\fhaCriticalityPre{#1}%
%}
%\define@key{fhaNumberKey}{fhaNumberKeyValue}{%
%    \def\fhaNumber{#1}%
%}
%\define@key{fhaCausesKey}{fhaCausesKeyValue}{%
%    \def\fhaCauses{#1}%
%}
%\define@key{fhaDescriptionKey}{fhaDescriptionKeyValue}{%
%    \def\fhaDescription{#1}%
%}
%\define@key{fhaMitigationKey}{fhaMitigationKeyValue}{%
%    \def\fhaMitigation{#1}%
%}
%\define@key{fhaCritPostKey}{fhaCritPostKeyValue}{%
%    \def\fhaCriticalityPost{#1}%
%}

\define@key{fhaKey}{fhaSystemKeyValue}{%
    \def\fhaSystem{#1}%
}
\define@key{fhaKey}{fhaFunctionKeyValue}{%
    \def\fhaFunction{#1}%
}
\define@key{fhaKey}{fhaCritPreKeyValue}{%
    \def\fhaCriticalityPre{#1}%
}
\define@key{fhaKey}{fhaNumberKeyValue}{%
    \def\fhaNumber{#1}%
}
\define@key{fhaKey}{fhaCausesKeyValue}{%
    \def\fhaCauses{#1}%
}
\define@key{fhaKey}{fhaDescriptionKeyValue}{%
    \def\fhaDescription{#1}%
}
\define@key{fhaKey}{fhaMitigationKeyValue}{%
    \def\fhaMitigation{#1}%
}
\define@key{fhaKey}{fhaCritPostKeyValue}{%
    \def\fhaCriticalityPost{#1}%
}



\makeatother

\newcommand\fhaTable[1][]{
% Grouping makes sure that later calls get the default values rather than
% the values from the last table
\begingroup 
% This parses the optional key-value parameters and runs the defined macros
% for each
\setkeys{fhaKey}{#1}
% Standard table with the macros used here
%\begin{table}
%    \centering
%    \begin{tabular}{c c c c c c c c c}
%    \hline
%    a & b & c & d & e & f & g & h  \\
%    \hline
%    \fhaSystem & \fhaFunction & \fhaCriticalityPre & \fhaNumber & \fhaCauses & \fhaDescription & \fhaMitigation & \fhaCriticalityPost \\
%    \end{tabular}
%\end{table}
\begin{table}[!ht]
\centering
\begin{tabular}{|p{0.18\textwidth}|p{0.18\textwidth}|p{0.18\textwidth}|p{0.18\textwidth}|}
%\Body 
\hline
\makecell[l]{\tiny{\textcolor{gray}{System:}} \\ \fhaSystem} & \makecell[l]{\tiny{\textcolor{gray}{Affected function:}} \\  \fhaFunction } & \makecell[l]{\tiny{\textcolor{gray}{Criticality:}} \\ \fhaCriticalityPre} &  \makecell[l]{\tiny{\textcolor{gray}{Number:}} \\ \fhaNumber }\\
\hline
\multicolumn{4}{|p{0.95\textwidth}|}{\makecell[l]{\tiny{\textcolor{gray}{Possible causes:}} \\ \begin{minipage}{0.95\textwidth} \fhaCauses \end{minipage}}} \\
\hline
\multicolumn{4}{|p{0.95\textwidth}|}{\makecell[l]{\tiny{\textcolor{gray}{Description:}} \\ \begin{minipage}{0.95\textwidth} \fhaDescription \end{minipage}}} \\
\hline
\multicolumn{4}{|p{0.95\textwidth}|}{\makecell[l]{\tiny{\textcolor{gray}{Mitigation:}} \\ \begin{minipage}{0.95\textwidth} \fhaMitigation \end{minipage}}} \\
\hline
\multicolumn{4}{|p{0.95\textwidth}|}{\makecell[l]{\tiny{\textcolor{gray}{Criticality with mitigation:}} \\ \begin{minipage}{0.95\textwidth} \fhaCriticalityPost \end{minipage}}} \\
\hline
\end{tabular}
\end{table}
% end of the group, replaces the changes to the key macros with default values
\endgroup
}



\begin{document}
\maketitle

\tableofcontents

\section{Introduction}
\label{sec:intro}
A Highline rescue is similar to one of a lead climber, a highly technical rescue. Furthermore, the patient is prone to suspension trauma due to long periods of hanging in a harness. Therefore, time is a priority.
Highlines span gaps, ridges, etc. Often, the anchors are only reachable by rappel/climbing. Thus, even if a terrestrial rescue transports the patient to one of the anchors, the way to a hospital is still long. Furthermore, the mode of transport to the hospital is probably a helicopter due to the terrain.\\

With longer Highlines, the rescue is escalating in complexity. Ropes need to be longer, and direct rappel off the Highline is no longer possible or dangerous (stretch in “static” ropes and webbings is not negligible and leads to large vertical movement during rappel). Furthermore, rescuers must pull the patient up for a rescue towards one of the anchors due to the Highline sag. \\

These thoughts lead the authors to discuss a helicopter-based rescue directly off the Highline with Heli Austria (namely Gabriel Falkner). A literature review showed that there is little published experience on this topic.\\

This document contains the thoughts of the authors on a helicopter-based highline rescue. The document shows the stages of the procedure in section \ref{sec:proc}, based on the procedure for an injured lead climber. Section \ref{sec:conventional} describes the conventional setup of a Highline. The authors derived a functional hazard analysis of the rescue procedure in section \ref{sec:fha}. Section \ref{sec:fha} also contains mitigation measures for the most critical risks. Section \ref{sec:modified} describes the resulting Highline setup for safer training. Finally, section \ref{sec:result} lists the procedure and the personal protective equipment in detail. The overall focus is on safety.

This document targets a rescue with a longline as a connection between rescuer and helicopter. Inferring from the contact of climbing ropes with the Highline, the authors suspect the textile longline rope with a diameter of approx. 20mm to have a negligible chance of cutting the Highline. If a thin steel hoist cable used for rotorcraft-based rescue poses a severe threat 


\section{Procedure of lead climber rescue as starting point}
\label{sec:proc}

This document starts with a procedure similar to rescuing a lead climber hanging in a harness from a rope. The load-bearing path defines the phases of this procedure. (The names of the phases are in bold font in the following list)

\begin{itemize}
\item  1st: \textbf{Approach and positioning} of the rotorcraft, including rescuer and rescue equipment above the patient.
\item  2nd: Rappel of rescuer and \textbf{attaching of the patient's harness} to the longline.
\item 3rd: \textbf{Climb of the rotorcraft} to transfer the patient's load off the climbing rope onto the longline.
\item 4th: \textbf{Cutting of the climbing rope} and check that the patient and rescuer are only connected to the rotorcraft.
\item 5th: \textbf{Departure} towards a free direction.
\end{itemize}

From the start of phase 2 until the end of phase 4, the rescue equipment, the harness of the patient, and the climbing rope connect the rotorcraft to the terrain. Nevertheless, the stretch and slack of the climbing rope allow movement in a limited range. \\

During phases 1 and 2, the load-bearing paths of the rescuer and the patient are entirely separate. During phase 3, the patient's load is gradually transferred to the longline. The transfer is complete if the climbing rope is free of load. After the climbing rope cut in phase 4, the rotorcraft carries the load of the rescuer and the patient. \\

This procedure works for a Highline rescue with a few minor modifications. These modifications result from the functional hazard analysis in chapter \ref{sec:fha}. The following shows the resulting procedure. Italics highlight differences between the procedure and the lead climber rescue procedure. 

\begin{itemize}
\item  1st: \textbf{Approach and positioning} of the rotorcraft, including rescuer and rescue equipment above the patient.
\item  2nd: Rappel of rescuer and \textbf{attaching of the patient's harness} to the longline.
\item 3rd: \textbf{Climb of the rotorcraft} to transfer the patient's load off the \textit{Highline} onto the longline. \textit{Furthermore, the rescuer positions the rotorcraft approx. 1m of horizontal distance to the Highline to check if any unwanted connection to the Highline exists}.
\item 4th: \textbf{Cutting of the} \textit{leash} and check that the patient and rescuer are only connected to the rotorcraft.
\item 5th: \textbf{Departure} towards a free direction.
\end{itemize}

\section{Conventional highline setup}
\label{sec:conventional}

Conventional Highlines consist of two anchors, a main line, one backup line, a leash ring, the leash, and the athlete's harness. Suitable points for anchors are trees, bolts, and removable equipment like cams, ice screws, or nuts. Anchors are generally redundant. \\
The main and backup lines are hung between the anchors. Personal preference, in combination with the used material, determines the pretension of the main line. The backup line prevents a fatality in case of a mainline failure due to falling debris or friction. It hangs loosely underneath the main line, often in loops. In case of a main line failure, the backup is an entirely redundant connection to the terrain. Nevertheless, this requires sufficient free horizontal distance underneath the line. \\
The main and backup lines go through a closed metal loop called the leash ring. The leash ring is an attachment point that can move along the line. The athlete connects the harness to the leash ring via a 1.2m long leash. This leash often consists of a climbing rope fed through a tubular webbing for redundancy.

\section{Functional hazard analysis and risk mitigation for a safe training setting}
\label{sec:fha}

Two outstanding dangers are present during all phases of the rescue:
\begin{itemize}
\item The backup line, a secondary line that runs parallel to the main line, is a crucial component in the rescue operation. It serves as a safety measure in case the main line fails. However, it's important to note that the backup line can also pose a potential danger of entanglement, as its loops can wrap around arms, legs, or material (on the harness, the longline rig, etc.). This can lead to complex problems that are difficult to solve. In a training environment, using a tubular material that both main line and backup go through is essential to eliminate the danger of entanglement.
\item In the event of a main line rupture, the line snaps back towards the anchors. This snapping action is more pronounced in what we refer to as ``high slack settings'', where there is a significant amount of slack in the line. To reduce the risk of line rupture, the slack of a training line is reduced by increasing the tension in the main line. This, in turn, increases the distance between the main line and the rotorcraft, minimizing the potential for the line to snap back towards the rotorcraft.
\end{itemize} 

The following Functional Hazard Analysis lists the risks of every procedure phase separately. It distinguishes between the functional systems \textit{rotorcraft}, \textit{longline}, \textit{harness}, \textit{leash}, \textit{rescuer}, and \textit{highline}. As the patient might be unconscious,   no functions are associated. 


The risks result from errors/inabilities of the listed systems during the rescue phases. This document lists the expected criticality of the identified risks in all phases of the rescue and proposes mitigation to create a safe training environment. 


The text uses \textit{highline} for the system composed of the two anchors, the main and backup line, and the leash ring as this system fulfills the functions \textbf{load bearing} and \textbf{psotitioning} as a single unit. Nevertheless, the risk lists a breakdown in the description if necessary. 


This document gives the criticality of a risk on three levels \textcolor{green}{low}, \textcolor{orange}{medium}, and \textcolor{red}{high}. The levels encode the complexity that a solution to the problem requires and the potential harm for rescuers, patients, and rotorcraft. Table \ref{tab:crit} gives the different levels.

\begin{table}[!ht]
\centering
\begin{tabular}{ c | c | c }
 criticality & complexity of solution & potential harm \\ 
 \hline
 \hline 
 \textcolor{green}{low} & no actions necessary & none \\ 
 \hline
 \textcolor{orange}{medium} & 1-2 simple steps, visual check, known communication & small bruises on arms or legs \\ 
 \hline
 \textcolor{red}{high} & \makecell{cutting, thinking thinking, \\ spontaneous problem solving/communication} & \makecell{everything more than \\ small bruises on arms or legs} \\  
   
\end{tabular}
\caption{Levels of criticality described by their complexity and potential harm.}
\label{tab:crit}
\end{table}

Table \ref{tab:func} gives the functions of the individual systems:
\begin{table}[!ht]
\centering
\begin{tabular}{ c | c  }
 system & function \\ 
 \hline
 \hline 
 rotorcraft & \makecell{load bearing \\ positioning (with help of rescuer)} \\
 \hline 
 rescuer & \makecell{positioning (self,rotatoric) \\ positioning (self,translatoric) \\ attaching patient to longline \\ cutting connection patient to highline} \\
 \hline
 longline & load bearing \\
 \hline
 highline & \makecell{load bearing \\ positioning} \\
 \hline
 leash & load bearing \\
 \hline
 harness (patient) & load bearing \\ 
\end{tabular}
\caption{Systems and respective functions.}
\label{tab:func}
\end{table}

The analysis does not differentiate between different causes of a failure of the rotorcraft to provide the functions \textbf{load bearing} and \textbf{positioning}. The rotorcaft corresponds to a black box system that can change the position in 3D space and lift a load (load bearing). In case of an emergency, the rotorcraft can drop the longline.

Humans perform some of the mitigation that is identified by the analysis. Therefore, the three roles \textcolor{teal}{rescuer}, \textcolor{purple}{patient}, and external \textcolor{blue}{observer} are defined. All humans involved in the training are connected via a shared radio.

\clearpage
\subsection{Phase 1: Approach and Positioning}
\label{sec:fha:phase1}

This phase starts when the systems rotorcraft+\textcolor{teal}{rescuer}+longline interacts with the systems \textcolor{purple}{patient}+highline. This includes aerodynamic interactions (e.g., downwash onto a highline). 

Risks that originate solely from using a rotorcraft+\textcolor{teal}{rescuer}+longline during the approach are \textbf{not} included here.
%1.1
\fhaTable[fhaSystemKeyValue=
			Rotorcraft,
		 fhaFunctionKeyValue=
		 	Load bearing,
	     fhaCritPreKeyValue=
	     	\textcolor{red}{high},
	     fhaNumberKeyValue=
	     	1.1,
	     fhaCausesKeyValue=
	     	Power or control deficit of rotorcraft,
	     fhaDescriptionKeyValue=
	     	{A power or control deficit of the rotorcraft during the positioning of the \textcolor{teal}{rescuer} relative 
	     	 to the \textcolor{purple}{patient} can occur, which causes the rotorcraft to lose altitude.  
	     	 This poses the risk of unintended contact between the longline and the highline.},
	     fhaMitigationKeyValue=
	     	{
	     	\tabitem abrasion-resistant tubing around highline and backup \\
	     	\tabitem Highline with abrasion-resistant ropes as backup \\
	     	\tabitem Approach of the rotorcraft to the highline from the side to which it can depart while autorotating at all times
	     	},
	     fhaCritPostKeyValue=
	     	{\textcolor{orange}{medium}, as the tubing prevents abrasion of main line and backup. 
	     	 With ropes as backups, it is less prone to damage from hitting objects. 
	     	 Approach from the departure side allows the rotorcraft to cancel the rescue and depart at all times.}
	     	]

%1.2
\fhaTable[fhaSystemKeyValue=
			Rotorcraft,
		 fhaFunctionKeyValue=
		 	Positioning,
	     fhaCritPreKeyValue=
	     	\textcolor{red}{high},
	     fhaNumberKeyValue=
	     	1.2,
	     fhaCausesKeyValue=
	     	{e.g. approaching to low/left/right relative to \textcolor{purple}{patient}, present turbulence, recirculation, etc.},
	     fhaDescriptionKeyValue=
	     	{During the positioning of the \textcolor{teal}{rescuer} relative to the \textcolor{purple}{patient} abrasive contact between longline and highline is possible. 
	     	This could damage the main line. Furthermore, the contact between longline and highline  might change the handling quality of the rotorcraft},
	     fhaMitigationKeyValue=
	     	{
	     	\tabitem abrasion-resistant tubing around highline and backup \\
	     	\tabitem Highline with abrasion-resistant ropes as backup \\
	     	\tabitem Canceling the rescue mission if the pilot or \textcolor{teal}{rescuer} have concerns
	     	},
	     fhaCritPostKeyValue=
	     	{\textcolor{orange}{medium}, as the tubing prevents abrasion of main line and backup.
	     	 With ropes as backups, it is less prone to damage from hitting objects. 
	     	 Approach from the departure side allows the rotorcraft to cancel the rescue and depart at all times.}
	     	]

%1.3
\fhaTable[fhaSystemKeyValue=
			\textcolor{teal}{Rescuer},
		 fhaFunctionKeyValue=
		 	Positioning (transl.),
	     fhaCritPreKeyValue=
	     	\textcolor{red}{high},
	     fhaNumberKeyValue=
	     	1.3,
	     fhaCausesKeyValue=
	     	{\textcolor{teal}{Rescuer} positions themself on the wrong side of the highline, thus preventing an 
	     	 unhindered departure of the rotorcraft.
	     	 Rappel device can get entangled in the highline or backup},
	     fhaDescriptionKeyValue=
	     	{During the rappel of the \textcolor{teal}{rescuer}, they could end on the side of the highline that does not correspond
	     	 to the departure direction of the rotorcraft, which prevents the rotorcraft from departing unhindered.
	     	  When rappelling too far, the longline might be pulled vertically along the line, which is discussed in 1.2},
	     fhaMitigationKeyValue=
	     	{
	     	\tabitem Communication between pilot and \textcolor{teal}{rescuer}, slow approach\\
	     	\tabitem Rappel with variable longline on the departure side of highline \\
	     	\tabitem \textcolor{purple}{Patient}, and external \textcolor{blue}{observer} intervene on radio in case of rappel errors
	     	},
	     fhaCritPostKeyValue=
	     	{\textcolor{orange}{medium}, as the approach to the \textcolor{purple}{patient} is similar to other maneuvers. 
	     	 \textcolor{purple}{Patient}, and external \textcolor{blue}{observer} can observe and intervene from different viewpoints  
	     	 \textcolor{teal}{Rescuer} can reach the rappel device and remove connections with the highline}
	     	]
	     	
%1.4
\fhaTable[fhaSystemKeyValue=
			\textcolor{teal}{Rescuer},
		 fhaFunctionKeyValue=
		 	Positioning (rot.),
	     fhaCritPreKeyValue=
	     	\textcolor{red}{high},
	     fhaNumberKeyValue=
	     	1.4,
	     fhaCausesKeyValue=
	     	{\textcolor{teal}{Rescuer} and \textcolor{purple}{patient} are in contact, wake of rotorcraft causes twisting of \textcolor{teal}{rescuer} and \textcolor{purple}{patient}.
	     	 This results in longline and leash being twisted, at extreme connecting both.
	     	 },
	     fhaDescriptionKeyValue=
	     	{During the contact of \textcolor{teal}{rescuer} and \textcolor{purple}{patient}, an unwanted twist of leash and longline could end in an unwanted connection of the rotorcraft to the highline.
	     	 Depending on the number of twists, untangling can require extensive work.
	     	 Furthermore, the twisted ropes can prevent further rappelling or lifting of the \textcolor{teal}{rescuer} relative to the \textcolor{purple}{patient}},
	     fhaMitigationKeyValue=
	     	{
	     	\tabitem \textcolor{teal}{Rescuer} grabs \textcolor{purple}{patient} with legs\\
	     	\tabitem Manually distancing leash and longline helps to prevent twisting \\
	     	\tabitem \textcolor{purple}{Patient}, and external \textcolor{blue}{observer} intervene on radio in case of dangerous twisting
	     	},
	     fhaCritPostKeyValue=
	     	{\textcolor{orange}{medium}, as the \textcolor{purple}{patient} can intervene and push the leash and longline apart, which prevents the twisting.
	     	}
	     	]

\clearpage

The functions ``Connecting patient's harness'' and ``Cutting the connection between patient and highline'' must not be executed during this rescue phase. Any attempt to do so must result in an immediate abort of the procedure.

%1.5
\fhaTable[fhaSystemKeyValue=
			{longline},
		 fhaFunctionKeyValue=
		 	{load transfer},
	     fhaCritPreKeyValue={
			\textcolor{red}{high}},
	     fhaNumberKeyValue=
	     	{1.5},
	     fhaCausesKeyValue=
	     	{Helicopter has to release longline}, 
	     fhaDescriptionKeyValue=
	     	{During the positioning of the \textcolor{teal}{rescuer} to the \textcolor{purple}{patient} by helicopter,
	      	it may be necessary for the helicopter to release the longline.
	       	This could put strain on the highline from the falling longline and \textcolor{teal}{rescuer}.
	       	The problem is that the highline can be subjected to multiple impacts by the falling \textcolor{teal}{rescuer} and longline.
	       	In addition, the weight of the \textcolor{teal}{rescuer} could pull the longline over the highline and cause damage.}, 
	     fhaMitigationKeyValue=
	     	{\tabitem Abrasion-resistant coating of line and backups \\
	     	\tabitem Line construction with multiple backups made of core-sheath ropes\\
	     	\tabitem Avoid having \textcolor{teal}{rescuer} and helicopters on different sides of the highline},
	     fhaCritPostKeyValue=
	     	{\textcolor{orange}{medium}, as the sheath prevents friction damage to the line. Damage caused by hitting can be mitigated by the backups used.}]

%1.6
\fhaTable[fhaSystemKeyValue=
			{Highline},
		 fhaFunctionKeyValue=
		 	{positioning},
	     fhaCritPreKeyValue=
	     	{\textcolor{red}{high}},
	     fhaNumberKeyValue=
	     	{1.6},
	     fhaCausesKeyValue=
	     	{Aerodynamic interaction of highline with helicopter downwash (rocking of the line in the wind)}, 
	     fhaDescriptionKeyValue=
	     	{Aerodynamic interaction between the helicopter and the highline is possible even before the \textcolor{teal}{rescuer} can reach the \textcolor{purple}{patient}.
	     	Since long highlines can swing strongly sideways and have little damping in this direction, this can prevent the helicopter from positioning the \textcolor{teal}{rescuer} with sufficient accuracy.
			Furthermore, the downwash can cause the loops of the highline backup to swing around and get caught in various places on the \textcolor{purple}{patient} or \textcolor{teal}{rescuer}.
			These unwanted connections have to be cut during the subsequent rescue process. If not released they could lead to unwanted tied-up situations.}, 
	     fhaMitigationKeyValue=
	     {\tabitem Termination of the attempt in case of acute swinging of the highline \\
	     \tabitem Hovering near highline to test aerodynamic interaction (careful approach)\\
	     \tabitem Sheathing the highline and backups to eliminate loops\\
	     \tabitem Higher highline tension than usual to reduce swinging},
	     fhaCritPostKeyValue=
	     {\textcolor{orange}{medium}, as the formation of loops is prevented and a slow approach to the final position of the helicopter allows an abort at any time}]
	     
%1.7
\fhaTable[fhaSystemKeyValue=
			{Highline},
		 fhaFunctionKeyValue=
		 	{load transfer},
	     fhaCritPreKeyValue=
	     	{\textcolor{red}{high}},
	     fhaNumberKeyValue=
	     	{1.7},
	     fhaCausesKeyValue=
	     	{Rockfall or other objects blown down by downwash}, 
	     fhaDescriptionKeyValue=
	     	{A falling stone, branch, carabiner, etc., is enough to cut the highline under tension.
	      	This causes the \textcolor{purple}{patient} to fall into the backup and hang lower than if hanging the highline.
	       	Several objects could fall down in a row and also cut off the backup that is subsequently loaded.}, 
	     fhaMitigationKeyValue=
	     {\tabitem Use of two core-sheath ropes as independent backups \\
	      \tabitem The double backup distributes the load parallel to both strands\\ 
	      \tabitem Cleaning the anchors of loose stones and objects that could be thrown up},
	     fhaCritPostKeyValue=
	     {\textcolor{orange}{medium}, since a sequential cutting of line and two impact-resistant core-shell ropes are considered very unlikely to be damaged by falling objects (since the cutting sensitivity also decreases with decreasing load)}]
	     
%1.8
\fhaTable[fhaSystemKeyValue=
			{Leash},
		 fhaFunctionKeyValue=
		 	{load transfer},
	     fhaCritPreKeyValue=
	     	{\textcolor{red}{high}},
	     fhaNumberKeyValue=
	     	{1.8},
	     fhaCausesKeyValue=
	     	{No causes for spontaneous leash rupture known}, 
	     fhaDescriptionKeyValue=
	     	{Spontaneous tearing of the leash due to contact with a sharp object under load.
	      	This is generally eliminated by the construction of a leash, which is made of climbing rope and enveloping tubular band with the same load capacity.
	       	However, it is included in the analysis because a simple leash can be used for easier cutting in a training scenario}, 
	     fhaMitigationKeyValue=
	     {\tabitem Avoid sharp objects on harnesses/clothing of \textcolor{teal}{rescuer} and \textcolor{purple}{patient} \\
	     \tabitem Additional self-belay loop that can be hung in the leash ring if the leash is damaged unintentionally \\
	     Leash shorter than the patient arm's length so that the leash ring can be reached while sitting in a harness},
	     fhaCritPostKeyValue=
	     {\textcolor{red}{high}, but very unlikely. Even with a simple leash when wearing suitable clothing and harnesses.}]

\clearpage

A failure of the patient's and rescuer's climbing harness will not be discussed here, as this is considered unlikely when
using an undamaged harnesses.


\subsection{Phase 2: Attaching the patient's harness}
\label{sec:fha:phase2}

%2.1
\fhaTable[fhaSystemKeyValue=
			{helicopter},
		 fhaFunctionKeyValue=
		 	{load transfer},
	     fhaCritPreKeyValue=
	     	{\textcolor{red}{high}},
	     fhaNumberKeyValue=
	     	{2.1},
	     fhaCausesKeyValue=
	     	{(Temporary) performance deficit of the helicopter}, 
	     fhaDescriptionKeyValue=
	     	{Before attaching the longline, this is described in 1.1. After attaching the longline (helicopter tied up), the rescuer's load can also be transferred to the highline via the connection between longline, \textcolor{purple}{patient}'s harness and leash if the helicopter accidentally loses altitude.
  			The longline can be unloaded and released if the helicopter has to leave its hovering position during this phase.
			Furthermore, the longline can be dropped and the \textcolor{teal}{rescuer} falls into the highline together with the \textcolor{purple}{patient}.
    		However, the falling longline can harm personnel and the highline (impact and friction on the highline).}, 
	     fhaMitigationKeyValue=
	     	{\tabitem Abrasion-resistant coating of highline and backups\\
	     	\tabitem \textcolor{purple}{patient} and \textcolor{teal}{rescuer} wear an appropriate helmet\\
	     	\tabitem Line construction with multiple backups made of core-sheath ropes\\
	     	\tabitem Sufficient fall height under the highline for line stretch by the weight of two persons\\
	     	\tabitem Terrestrial rescue equipment and an easily accessible highline anchor\\
	     	\tabitem Sufficient personnel with knowledge at the anchors for a terrestrial rescue},
	     fhaCritPostKeyValue=
	     	{\textcolor{red}{high}. The sheathing prevents friction damage to the line.
	      	Damage caused by impact can be mitigated by the backups used.
 			Injury to people caused by falling ropes cannot be ruled out, but immediate, rapid rescue can be prepared.
			The helmet offers head (and eye) protection. A sufficient fall height can be planned for when setting up.}]

%2.2
\fhaTable[fhaSystemKeyValue=
			{helicopter},
		 fhaFunctionKeyValue=
		 	{positioning},
	     fhaCritPreKeyValue=
	     	{\textcolor{red}{high}},
	     fhaNumberKeyValue=
	     	{2.2},
	     fhaCausesKeyValue=
	     	{Tied-up helicopter steers the highline due to position fluctuations}, 
	     fhaDescriptionKeyValue=
	     	{During the positioning of the \textcolor{teal}{rescuer} to the \textcolor{purple}{patient} by the helicopter, relative movement with contact between the longline and the highline could occur for various reasons.
	      	This could cause friction damage to the highline.
	      	Furthermore, the contact between the line and the longline could affect the control behavior of the helicopter.
	      	In the most serious case, the longline could have to be released (in this case, see 2.1)}, 
	     fhaMitigationKeyValue=
	     	{\tabitem Abrasion-resistant coating of line and backups\\
	     	\tabitem Line construction with multiple backups made of core-sheath ropes\\},
	     fhaCritPostKeyValue=
	     	{\textcolor{orange}{medium}, as the sheath prevents friction damage to the line. Damage caused by hitting can be mitigated by the backups used.
 			Releasing the longline does not cause the \textcolor{teal}{rescuer} to fall}]
 
%2.3 
\fhaTable[fhaSystemKeyValue=
			{\textcolor{teal}{rescuer}},
		 fhaFunctionKeyValue=
		 	{positioning (transl.)},
	     fhaCritPreKeyValue=
	     	{\textcolor{red}{high}},
	     fhaNumberKeyValue=
	     	{2.3},
	     fhaCausesKeyValue=
	     	{Flight \textcolor{teal}{rescuer} positions himself too far below \textcolor{purple}{patient} to attach \textcolor{purple}{patient} to rope.}, 
	     fhaDescriptionKeyValue=
	     	{Immediately after attaching the patient, the \textcolor{teal}{rescuer} can lower himself too far with the variable longline or hang below the \textcolor{purple}{patient} due to the helicopter losing altitude.
	       	This can be so much that the longline is unloaded and \textcolor{purple}{patient} and \textcolor{teal}{rescuer} hang vertically above each other.
	        In this state, twisting of the longline and leash can no longer be effectively prevented.}, 
	     fhaMitigationKeyValue=
	     	{\tabitem Slow lowering on the variable longline to avoid ending too far below the patient \\ 
			\tabitem Early warning of the pilot about any unintentional descent of the helicopter\\
			\tabitem \textcolor{blue}{observer} and \textcolor{purple}{patient} intervene early by warning on the radio (air \textcolor{teal}{rescuer} too low)\\
			\tabitem If the longline is unloaded, the \textcolor{purple}{patient} grabs the highline to prevent twisting.},
	     fhaCritPostKeyValue=
	     {\textcolor{orange}{medium}, as it can be prevented by the \textcolor{purple}{patient} with a short leash (highline within arm's reach).}]

%2.4
\fhaTable[fhaSystemKeyValue=
			{\textcolor{teal}{rescuer}},
		 fhaFunctionKeyValue=
		 	{positioning (rot.)},
	     fhaCritPreKeyValue=
	     	{\textcolor{red}{high}},
	     fhaNumberKeyValue=
	     	{2.4},
	     fhaCausesKeyValue=
	     	{The \textcolor{teal}{rescuer} and \textcolor{purple}{patient} are in contact, the downwash from the helicopter causes them to twist
	     	and thus, the rope and leash can get twisted if the \textcolor{teal}{rescuer} does not take adequate action.}, 
	     fhaDescriptionKeyValue=
	     	{Even if the \textcolor{teal}{rescuer} is not positioned too low below the \textcolor{purple}{patient}, the leash and rope can become twisted.
	      	During the rescue, the flight \textcolor{teal}{rescuer} should prevent this to avoid an unwanted restraint situation.}, 
	     fhaMitigationKeyValue=
	     	{\tabitem Air \textcolor{teal}{rescuer} safely grasps \textcolor{purple}{patient} with legs in case of contact\\
	     	\tabitem Twisting can be made more difficult by the \textcolor{teal}{rescuer} pushing the leash and rope apart (approx. 50cm).\\
	     	\tabitem After the \textcolor{purple}{patient} has been safely attached to the rope, the helicopter can create a horizontal distance of approximately one leash length between the highline and the rope.
	     	This aides in identifying unwanted connections and effectively prevents twisting of longline and leash.\\
	     	\tabitem \textcolor{blue}{observer} and \textcolor{purple}{patient} intervene early via radio warning},
	     fhaCritPostKeyValue=
	     {\textcolor{green}{low}, as the \textcolor{purple}{patient} can intervene early, and twisting can be stopped by the \textcolor{purple}{patient} grabbing the highline.}]
	     
%2.5
\fhaTable[fhaSystemKeyValue=
			{\textcolor{teal}{rescuer}},
		 fhaFunctionKeyValue=
		 	{Attaching the \textcolor{purple}{patient} harness in longline},
	     fhaCritPreKeyValue=
	     	{\textcolor{orange}{medium}},
	     fhaNumberKeyValue=
	     	{2.5},
	     fhaCausesKeyValue=
	     	{\textcolor{teal}{rescuer} hangs too low/high or too far to the side of the \textcolor{purple}{patient} to attach his harness to the rope}, 
	     fhaDescriptionKeyValue=
	     	{During the attachment phase, various readjustments of the position of the air \textcolor{teal}{rescuer} may be necessary to attach the \textcolor{purple}{patient}.
	      	This increases the time and, thus, the risk in this phase.
	      	Furthermore, the patient's harness may be attached incorrectly due to the lack of visibility on the harness/leash/line/etc. }, 
	     fhaMitigationKeyValue=
	     	{\textcolor{purple}{patient} checks with \textcolor{teal}{rescuer} whether he is correctly hooked in and communicates all related problems on the radio},
	     fhaCritPostKeyValue=
	     	{\textcolor{green}{low}, since the \textcolor{purple}{patient} checks the situation by the four eyes principle. In case of a security risk, feedback is given, and appropriate action is taken}]
	     
\clearpage
During this phase, the air \textcolor{teal}{rescuer} is not allowed to carry out the function “disconnecting the patient from the highline”. If the timing is incorrect, the \textcolor{purple}{patient} intervenes by issuing a warning on the radio.
Since cases 2.1 and 2.2 already deal with the release of the rope in this phase, it will not be explicitly listed again here.


%2.6
\fhaTable[fhaSystemKeyValue=
			{Highline},
		 fhaFunctionKeyValue=
		 	{positioning},
	     fhaCritPreKeyValue=
	     	{\textcolor{red}{high}},
	     fhaNumberKeyValue=
	     	{2.6},
	     fhaCausesKeyValue=
	     	{Aerodynamic interaction of highline with helicopter downwash (rocking of the line in the wind)}, 
	     fhaDescriptionKeyValue=
	     	{Aerodynamic interaction between the helicopter and the highline is possible even before the \textcolor{teal}{rescuer} can reach the \textcolor{purple}{patient}.
	      	Since long highlines can swing strongly sideways and have little damping in this direction, this can prevent the helicopter from positioning the \textcolor{teal}{rescuer} with sufficient accuracy.
	       	Furthermore, the downwash can cause the loops of the highline backup to swing around and get caught in various places on the \textcolor{purple}{patient} or \textcolor{teal}{rescuer}.
	        These could have to be cut during the subsequent rescue process or could lead to unwanted tied-up situations.}, 
	     fhaMitigationKeyValue=
	     	{\tabitem Test for upward-swinging behavior already in phase 1 (approach)\\
	     	\tabitem Sheathing of line and backup to prevent loops in a training session},
	     fhaCritPostKeyValue=
	     	{\textcolor{orange}{medium}, as the formation of loops is prevented, and a swinging up in phase 1 can be tested by a slow approach}]

%2.7
\fhaTable[fhaSystemKeyValue=
			{Highline},
		 fhaFunctionKeyValue=
		 	{load transfer},
	     fhaCritPreKeyValue=
	     	{\textcolor{red}{high}},
	     fhaNumberKeyValue=
	     	{2.7},
	     fhaCausesKeyValue=
	     	{Rockfall or other objects blown down by downwash}, 
	     fhaDescriptionKeyValue=
	     	{A falling stone, branch, carabiner, etc., is enough to cut the line under tension.
	      	As a result, the \textcolor{purple}{patient} suddenly falls into the longline and the line no longer supports his weight.
	      	This can lead to a high control load for the pilot.
	      	The length of the connection between the rope and the patient harness limits the height of the fall}, 
	     fhaMitigationKeyValue=
	     	{\tabitem Cleaning the anchors of loose stones and objects that could be thrown up \\
	     	\tabitem Prohibit actions near the highline that involve throwing/kicking off objects/etc. \\
	     	\tabitem Minimize the possible fall height of the \textcolor{purple}{patient} in the event of a highline tear},
	     fhaCritPostKeyValue=
	     	{\textcolor{orange}{medium}. It is very unlikely if the anchors are cleaned and personnel at the anchors are careful.
			In addition, the fall height can be kept small, and the pilot can be prepared}]

%2.8
\fhaTable[fhaSystemKeyValue=
			{Leash},
		 fhaFunctionKeyValue=
		 	{load transfer},
	     fhaCritPreKeyValue=
	     	{\textcolor{red}{high}},
	     fhaNumberKeyValue=
	     	{2.8},
	     fhaCausesKeyValue=
	     	{Fall of two people into the leash when releasing the longline}, 
	     fhaDescriptionKeyValue=
	     	{If the longline has to be released during this phase, the \textcolor{teal}{rescuer} and \textcolor{purple}{patient} will fall into the patient's leash.}, 
	     fhaMitigationKeyValue=
	     	{\tabitem Avoid sharp objects on harnesses/clothing of \textcolor{teal}{rescuer} and \textcolor{purple}{patient}\\
	     	\tabitem Cutting tool for the leash cut remains secured in a holder\\
	     	\tabitem No unprotected, sharp objects on \textcolor{purple}{patient} and \textcolor{teal}{rescuer}},
	     fhaCritPostKeyValue=
	     	{\textcolor{red}{high}, but can easily be avoided by taking measures (use of suitable clothing and harness and appropriate caution).}]

%2.9
\fhaTable[fhaSystemKeyValue=
			{harness},
		 fhaFunctionKeyValue=
		 	{load transfer},
	     fhaCritPreKeyValue=
	     	{\textcolor{orange}{medium}},
	     fhaNumberKeyValue=
	     	{2.9},
	     fhaCausesKeyValue=
	     	{Fall of two people in the leash when releasing the longline}, 
	     fhaDescriptionKeyValue=
	     	{If the longline has to be released in this phase, the \textcolor{teal}{rescuer} and \textcolor{purple}{patient} fall into the patient's leash.
	      	The rescuer's load path is via the longline into the belay loop of the patient's harness.
	       	The leash is usually tied parallel to the belay ring (threaded through the loops of the harness, similarly to tying in a climbing rope).}, 
	     fhaMitigationKeyValue=
	     	{\tabitem Attach the longline to the leash attachment loop and the patient harness attachment ring.\\
	      	If only one is possible, attach it to the leash attachment loop.\\
	     	\tabitem \textcolor{purple}{patient} checks that the \textcolor{teal}{rescuer} has correctly attached the longline\\},
	     fhaCritPostKeyValue=
	     	{\textcolor{green}{low}. Can be controlled very well through communication and the 4-eyes principle.\\
 			However, a suitable communication channel is a mandatory}]

\clearpage
\subsection{Phase 3: Climb of the rotorcraft to transfer load}
\label{sec:fha:phase3}

This phase begins after the \textcolor{purple}{patient} has been successfully hooked into the longline and the \textcolor{teal}{rescuer} and \textcolor{purple}{patient} are lifted
by the helicopter. The lifting is complete when the highline is (almost) free of load.


%3.1
\fhaTable[fhaSystemKeyValue=
			{helicopter},
		 fhaFunctionKeyValue=
		 	{load transfer},
	     fhaCritPreKeyValue=
	     	{\textcolor{red}{high}},
	     fhaNumberKeyValue=
	     	{3.1},
	     fhaCausesKeyValue=
	     	{(Temporary) performance deficit of the helicopter}, 
	     fhaDescriptionKeyValue=
	     	{After attaching the longline (helicopter tied up), the rescuer's load can be transferred to the highline via the longline, patient harness, and leash if the helicopter accidentally loses altitude.
	      	The load is gradually transferred from the highline.
	       	However, if the helicopter abruptly stops transferring the load, the highline can swing vertically.
	       	In this case, the longline could be periodically loaded and unloaded.
	       	However, since the \textcolor{purple}{patient} and \textcolor{teal}{rescuer} are still secured via the highline, releasing the longline does not lead to a person falling.}, 
	     fhaMitigationKeyValue=
	     	{\tabitem Abrasion-resistant coating of line and backups \\
	     	\tabitem \textcolor{purple}{patient} and \textcolor{teal}{rescuer} wear an appropriate helmet\\
	     	\tabitem Line construction with multiple backups made of core-sheath ropes\\
	     	\tabitem Sufficient fall height under the highline for a fall with the weight of two persons\\
	     	\tabitem Terrestrial rescue equipment and an easily accessible highline anchor \\
	     	\tabitem Sufficient personnel with knowledge at the anchors for a terrestrial rescue \\
	     	\tabitem Increased preload in the highline to keep the fall height low \\
	     	\tabitem Pilot's awareness\\},
	     fhaCritPostKeyValue=
	     	{\textcolor{red}{high}. The sheathing prevents friction damage to the line.
			Damage caused by impact can be mitigated by the backups used.
			Injury to people caused by the falling rope cannot be ruled out, but immediate, rapid rescue can be prepared.
			The helmet offers head protection. A sufficient fall height can be planned for when setting up}]
	     
%3.2
\fhaTable[fhaSystemKeyValue=
			{helicopter},
		 fhaFunctionKeyValue=
		 	{positioning},
	     fhaCritPreKeyValue=
	     	{\textcolor{red}{high}},
	     fhaNumberKeyValue=
	     	{3.2},
	     fhaCausesKeyValue=
	     	{Tied-up helicopter steers the highline due to fluctuations in its hover position}, 
	     fhaDescriptionKeyValue=
	     	{During the lifting of the \textcolor{teal}{rescuer} and \textcolor{purple}{patient} by the helicopter, relative movement with contact between the rope and the highline could occur for various reasons.
	      	This could cause friction damage to the Highline.
	       	Furthermore, the contact between the highline and the longline could affect the control behavior of the helicopter.
	       	In the worst case, the longline needs to be released (in this case, see 2.1)}, 
	     fhaMitigationKeyValue=
	     	{\tabitem Abrasion-resistant coating of line and backups\\
	     	\tabitem Line construction with multiple backups made of core-sheath ropes},
	     fhaCritPostKeyValue=
	     	{\textcolor{orange}{medium}, as the sheath prevents friction damage to the line.
	     	Damage caused by hitting can be mitigated by the backups used.
	     	Dropping the rope does not cause the \textcolor{teal}{rescuer} to fall}]

%3.3
\fhaTable[fhaSystemKeyValue=
			{\textcolor{teal}{rescuer}},
		 fhaFunctionKeyValue=
		 	{positioning (transl.)},
	     fhaCritPreKeyValue=
	     	{\textcolor{red}{high}},
	     fhaNumberKeyValue=
	     	{3.3},
	     fhaCausesKeyValue=
	     	{Highline gets between \textcolor{teal}{rescuer} and \textcolor{purple}{patient} or the rappel device on the longline gets caught in the highline}, 
	     fhaDescriptionKeyValue=
	     	{When the helicopter lifts the \textcolor{purple}{patient} and \textcolor{teal}{rescuer}, the highline can get between the \textcolor{purple}{patient} and the \textcolor{teal}{rescuer}.
	      	In this case, free departure is not possible even after cutting the leash (next phase).
	       	If the highline gets stuck on the rappel device in the longline, a complex load condition of the longline, leash, and other components occurs.}, 
	     fhaMitigationKeyValue=
	     	{\tabitem This situation can be easily prevented by creating a lateral distance of approximately one leash length between the rope and the highline before starting the lifting of patient and rescuer by the helicopter\\
	     	\tabitem \textcolor{blue}{observer} and \textcolor{purple}{patient} intervene early by warning on the radio (air \textcolor{teal}{rescuer} too low)\\
	     	\tabitem If the situation cannot be resolved by the \textcolor{teal}{rescuer}, the \textcolor{purple}{patient} intervenes and lifts the highline above himself or the \textcolor{teal}{rescuer}.\\
	       	If the rappel system threatens to get caught on the highline, the \textcolor{purple}{patient} pushes the rope and highline apart},
	     fhaCritPostKeyValue=
	     	{\textcolor{green}{low}, since it can be subsequently released by the flight \textcolor{teal}{rescuer} or \textcolor{purple}{patient} using a short leash (highline within grasping distance)}]

%3.4
\fhaTable[fhaSystemKeyValue=
			{\textcolor{teal}{rescuer}},
		 fhaFunctionKeyValue=
		 	{positioning (rot.)},
	     fhaCritPreKeyValue=
	     	{\textcolor{red}{high}},
	     fhaNumberKeyValue=
	     	{3.4},
	     fhaCausesKeyValue=
	     	{The \textcolor{teal}{rescuer} and \textcolor{purple}{patient} are in contact, the downwash from the helicopter causes them to twist and thus the longline and leash to twist.
 			The \textcolor{teal}{rescuer} does not take adequate action.}, 
	     fhaDescriptionKeyValue=
	     	{Even if the \textcolor{teal}{rescuer} is not positioned too low below the \textcolor{purple}{patient}, the leash and rope can become twisted.
	      	During the maneuver, the flight \textcolor{teal}{rescuer} should prevent this to avoid an unwanted restraint situation of the helicopter.}, 
	     fhaMitigationKeyValue=
	     	{\tabitem By positioning the helicopter approximately one leash length in horizontal distance between line and longline, unwanted connections become apparent and twisting is prevented\\
	     	\tabitem If there is no reaction from the \textcolor{teal}{rescuer}, the \textcolor{purple}{patient} grabs the highline and prevents twisting\\
	     	\tabitem \textcolor{purple}{patient} checks rope and highline for possible entanglement when lifting and warns early\\
	     	\tabitem \textcolor{blue}{observer} and \textcolor{purple}{patient} intervene early via radio warning\\},
	     fhaCritPostKeyValue=
	     	{\textcolor{green}{low}, as the \textcolor{purple}{patient} can intervene early and prevent twisting by grasping the highline.}]

\clearpage
The flight rescuer's "hooking the patient harness into the rope" function is no longer relevant in this and all subsequent
phases. The function of "disconnecting the patient from the highline" must not be carried out in this phase, as otherwise,
the pre-tension in the highline will generate a strong vertical oscillation that can lead to entanglement in the rope,
\textcolor{purple}{patient}, flight \textcolor{teal}{rescuer}, or helicopter and must be avoided at all costs. If the flight \textcolor{teal}{rescuer} initiates the next phase (cutting
the leash) too early, the \textcolor{purple}{patient} will intervene with an early warning.\\


Since cases 3.1 and 3.2 already deal with the release of the rope in this phase, it will not be explicitly listed again here.
	     	     
	     
%3.5
\fhaTable[fhaSystemKeyValue=
			{Highline},
		 fhaFunctionKeyValue=
		 	{positioning},
	     fhaCritPreKeyValue=
	     	{\textcolor{red}{high}},
	     fhaNumberKeyValue=
	     	{3.5},
	     fhaCausesKeyValue=
	     	{Aerodynamic interaction of highline with helicopter downwash (rocking of the line in the wind)}, 
	     fhaDescriptionKeyValue=
	     	{By gradually relieving the load on the highline, the preload and thus the vibration behavior changes.
	      	This can cause the line to swing due to interaction with the downwash during lifting of patient and rescuer. This can occur even though no swinging was observed in the previous phases.}, 
	     fhaMitigationKeyValue=
	      	{\tabitem If the swinging is too strong: Lower the \textcolor{teal}{rescuer} and \textcolor{purple}{patient} to return to a stable situation and abort the rescue.\\
	     	\tabitem Hand of \textcolor{teal}{rescuer} on highline for guiding and manually dampening the vibration \\
	     	\tabitem Short leash to ensure that the highline can be reached by the \textcolor{teal}{rescuer}\\
	     	\tabitem Sheathing of line and backup to prevent loops\\},
	     fhaCritPostKeyValue=
	     	{\textcolor{orange}{medium}, as the formation of loops is prevented and the relief takes place gradually.
 			This means that any swinging can be noticed early (haptically) and can be actively dampened to a certain extent by the arm of the \textcolor{teal}{rescuer}.
			In the case of uncontrollable swinging, the highline can be loaded from the anchors to return to a stable state.}]
	     
%3.6
\fhaTable[fhaSystemKeyValue=
			{Highline},
		 fhaFunctionKeyValue=
		 	{load transfer},
	     fhaCritPreKeyValue=
	     	{\textcolor{red}{high}},
	     fhaNumberKeyValue=
	     	{3.6},
	     fhaCausesKeyValue=
	     	{Rockfall or other objects blown down by downwash}, 
	     fhaDescriptionKeyValue=
	     	{A falling stone, branch, carabiner, etc., is enough to cut the line under tension. As a result, the highline's load is suddenly transferred to the helicopter.
	      	This can lead to a high control load for the pilot.}, 
	     fhaMitigationKeyValue=
	     	{\tabitem Cleaning the anchors of loose stones and objects that could be thrown up\\
	     	\tabitem Prohibit actions near the highline that involve throwing/kicking off objects/etc.\\
	     	\tabitem Include pilot awareness/assessment},
	     fhaCritPostKeyValue=
	     	{\textcolor{orange}{medium}. It is very unlikely that the anchors are cleaned sufficiently and the staff at the anchors behave accordingly. In addition, the pilot can be prepared}]

%3.7
\fhaTable[fhaSystemKeyValue=
			{Leash},
		 fhaFunctionKeyValue=
		 	{load transfer},
	     fhaCritPreKeyValue=
	     	{\\textcolor{red}{high}},
	     fhaNumberKeyValue=
	     	{3.7},
	     fhaCausesKeyValue=
	     	{Fall of two people into the leash when releasing the longline}, 
	     fhaDescriptionKeyValue=
	     	{If the longline has to be released during this phase, the \textcolor{teal}{rescuer} and \textcolor{purple}{patient} will fall into the patient's leash.
	      	The highest fall load occurs at the highest point of the lifting process. This is the largest nominal load for the leash.}, 
	     fhaMitigationKeyValue=
	     	{\tabitem Avoid lifting above the zero line (unloaded highline). This can be easily detected by the \textcolor{teal}{rescuer} due to the mobility of the leash ring \\
	     	\tabitem Avoid sharp objects on harnesses/clothing of \textcolor{teal}{rescuer} and \textcolor{purple}{patient}\\
	     	\tabitem Cutting tool for the leash cut remains secured in a holder\\
	     	\tabitem No unprotected, sharp objects on \textcolor{purple}{patient} and \textcolor{teal}{rescuer}},
	     fhaCritPostKeyValue=
	     	{\textcolor{red}{high}, but can easily be avoided by taking measures (use of suitable clothing and harness and appropriate caution).}]
	     
%3.8
\fhaTable[fhaSystemKeyValue=
			{harness},
		 fhaFunctionKeyValue=
		 	{load transfer},
	     fhaCritPreKeyValue=
	     	{\textcolor{orange}{medium}},
	     fhaNumberKeyValue=
	     	{3.8},
	     fhaCausesKeyValue=
	     	{Fall of two people in the leash when releasing the rope}, 
	     fhaDescriptionKeyValue=
	     	{If the rope has to be released in this phase, the \textcolor{teal}{rescuer} and \textcolor{purple}{patient} fall into the patient's leash.
	      	The rescuer's load path is via the longline into the belay loop of the patient's harness.
	       	The leash is usually tied parallel to the belay ring (threaded through the loops of the harness part and the leg part of the harness, as if directly tying in a climbing rope).}, 
	     fhaMitigationKeyValue=
	     {As in phase 2: Hook the longline into the leash's attachment loop and the patient harness's attachment ring.
	      If only one is possible, Hook it into the leash's attachment loop.},
	     fhaCritPostKeyValue=
	     	{\textcolor{green}{low}. It can be controlled very well through communication and the 4-eyes principle.
	     	However, a suitable communication channel is a prerequisite.}]

\clearpage
\subsection{Phase 4: Cutting the Leash}
\label{sec:fha:phase4}
This phase begins with a tool being used to cut the leash. Errors that arise from direct problems when cutting
the leash are listed in the leash section of this chapter.

%4.1
\fhaTable[fhaSystemKeyValue=
			{helicopter},
		 fhaFunctionKeyValue=
		 	{load transfer},
	     fhaCritPreKeyValue=
	     	{\textcolor{red}{high}},
	     fhaNumberKeyValue=
	     	{4.1},
	     fhaCausesKeyValue=
	     	{(Temporary) performance deficit of the helicopter}, 
	     fhaDescriptionKeyValue=
	     	{An unintentional loss of altitude during the cutting process of the leash can lead to an unintentional load on a leash that has already been cut.
	      	From this point on, the highline is no longer a fall protection device for \textcolor{teal}{rescuer} and \textcolor{purple}{patient}.}, 
	     fhaMitigationKeyValue=
	     	{\tabitem Leash shorter than arm length to ensure accessibility \\
	     	\tabitem Stay in this phase as short as possible -> quick, safe cut\\
	     	\tabitem \textcolor{purple}{patient} with a second pair of scissors ready to help out if necessary\\
	     	\tabitem Cut resistance of the sheath on the highline and the backups to prevent unwanted damage to theses systems\\
	     	\tabitem The \textcolor{purple}{patient} has a self-belay loop that can be hooked into the leash ring if necessary, and the highline functions as a safe fall protection device.\\
	     	\tabitem Sufficient height under the highline for a fall with stretch of the highline by the weight of two persons \\
	     	\tabitem Terrestrial rescue equipment and an easily accessible highline anchor \\
	     	\tabitem Sufficient personnel with knowledge at the anchors for a terrestrial rescue},
	     fhaCritPostKeyValue=
	     	{\textcolor{red}{high}. A problem with the helicopter during this phase can have fatal consequences if it occurs unannounced.
 			With a bit of time, a safe connection to the highline can be re-established}]
	     
%4.2
\fhaTable[fhaSystemKeyValue=
			{helicopter},
		 fhaFunctionKeyValue=
		 	{positioning},
	     fhaCritPreKeyValue=
	     	{\textcolor{orange}{medium}},
	     fhaNumberKeyValue=
	     	{4.2},
	     fhaCausesKeyValue=
	     	{Tied-up helicopter steers the highline due to fluctuations in the hover position}, 
	     fhaDescriptionKeyValue=
	     	{The unloaded highline can easily be deflected in all directions. However, it swings back to its zero position after the leash is cut.
	     	This could cause it to get caught on the \textcolor{teal}{rescuer} or \textcolor{purple}{patient}}, 
	     fhaMitigationKeyValue=
	     	{\tabitem Abrasion-resistant coating of line and backups to avoid loops\\
	     	\tabitem As little lateral deflection of the highline as possible when lifted by the helicopter},
	     fhaCritPostKeyValue=
	     	{\textcolor{orange}{medium}. The sheath makes tangling less likely, and only one strand needs to be checked for entanglements.}]

%4.3
\fhaTable[fhaSystemKeyValue=
			{\textcolor{teal}{rescuer}},
		 fhaFunctionKeyValue=
		 	{positioning (transl.)},
	     fhaCritPreKeyValue=
	     	{\textcolor{red}{high}},
	     fhaNumberKeyValue=
	     	{4.3},
	     fhaCausesKeyValue=
	     	{Flight \textcolor{teal}{rescuer} and \textcolor{purple}{patient} are close to the line, thus entanglement with the backup and highline is possible. The resulting situation might be unsafe and hard to solve.}, 
	     fhaDescriptionKeyValue=
	     	{Before cutting the leash, the \textcolor{purple}{patient} and the flight \textcolor{teal}{rescuer} should be approximately one leash length away from the highline (to the side).
	     	This makes it easy to see whether there is currently a second connection between the \textcolor{teal}{rescuer} or the \textcolor{purple}{patient} besides of the leash.
	      	If this is the case, this connection must be broken before cutting the leash.
	       	If the \textcolor{teal}{rescuer} does not create such a distance by positioning the helicopter, the risk of becoming entangled after the cut of the leash increases.}, 
	     fhaMitigationKeyValue=
	     	{\tabitem This situation can be easily prevented by creating a lateral distance of approximately one leash length between the rope and the highline before starting to lift.\\
	     	\tabitem \textcolor{blue}{observer} and \textcolor{purple}{patient} intervene early by giving a warning on the radio if no lateral distance has been created by the \textcolor{teal}{rescuer} or if an unwanted second connection exists.\\
			\tabitem  If a second connection exists and cannot be released by the flight \textcolor{teal}{rescuer}, the \textcolor{purple}{patient} intervenes and releases the second connection in consultation with the flight \textcolor{teal}{rescuer} (if possible).},
	     fhaCritPostKeyValue=
	     	{\textcolor{green}{low}, since the lateral distance to the highline provides a good overview of the existing connections and a pair of scissors for the \textcolor{purple}{patient} and \textcolor{teal}{rescuer} creates a large area in which unwanted connections can be cut.}]

%4.4
\fhaTable[fhaSystemKeyValue=
			{\textcolor{teal}{rescuer}},
		 fhaFunctionKeyValue=
		 	{positioning (rot.)},
	     fhaCritPreKeyValue=
	     	{\textcolor{red}{high}},
	     fhaNumberKeyValue=
	     	{4.4},
	     fhaCausesKeyValue=
	     	{Downwash can twist \textcolor{teal}{rescuer} and \textcolor{purple}{patient}, which can result in a tied-up situation even after a cut of the leash (friction of the leash on the rope due to wrapping)}, 
	     fhaDescriptionKeyValue=
	     	{If the leash is twisted or caught in the rappel device of the variable longline, the helicopter is tied up even after the leash has been cut.
	      	These connections can be difficult or even impossible to release due to tension.}, 
	     fhaMitigationKeyValue=
	     	{\tabitem By positioning the helicopter approximately one leash length in horizontal distance between line and longline the visibility of unwanted connections is increased, and twisting is effectively prevented \\
	     	\tabitem If there is no reaction from the \textcolor{teal}{rescuer}, the \textcolor{purple}{patient} grabs the highline and prevents twisting (the leash length is adjusted accordingly)\\
			\tabitem \textcolor{blue}{observer} and \textcolor{purple}{patient} intervene early via radio warning},
	     fhaCritPostKeyValue=
	     	{\textcolor{green}{low}, as the \textcolor{purple}{patient} can intervene early and prevent twisting by grasping the highline.}]
	     
%4.5
\fhaTable[fhaSystemKeyValue=
			{\textcolor{teal}{rescuer}},
		 fhaFunctionKeyValue=
		 	{dissolving the connection Patient with Highline},
	     fhaCritPreKeyValue=
	     	{\textcolor{red}{high}},
	     fhaNumberKeyValue=
	     	{4.5},
	     fhaCausesKeyValue=
	     	{The flight \textcolor{teal}{rescuer} cuts the leash before all other (unwanted) connections of the \textcolor{purple}{patient} to the highline are cut.}, 
	     fhaDescriptionKeyValue=
	     	{For example, by looping a highline backup around the patient's foot, the helicopter would remain tied up.
	     	Furthermore, this connection could be complicated to release after the leash has been cut}, 
	     fhaMitigationKeyValue=
	     	{\tabitem By positioning the helicopter approximately one leash length horizontally between the line and the longline, the visibility of unwanted connections is increased.\\
	     	\tabitem \textcolor{purple}{patient} and \textcolor{blue}{observer} check whether \textcolor{purple}{patient}, \textcolor{teal}{rescuer} or rope have unwanted connections before \textcolor{teal}{rescuer} reaches for the scissors\\
	     	\tabitem \textcolor{blue}{observer} and \textcolor{purple}{patient} intervene early via radio warning},
	     fhaCritPostKeyValue=
	     	{\textcolor{orange}{medium}, as 6 eyes check for unwanted connections. At the same time, this phase of the rescue process should be deliberately kept short.
			This requires early start of the check by \textcolor{purple}{patient} and \textcolor{blue}{observer} and rapid warning.}]

%4.6
\fhaTable[fhaSystemKeyValue=
			{longline},
		 fhaFunctionKeyValue=
		 	{load transfer},
	     fhaCritPreKeyValue=
	     	{\textcolor{red}{high}},
	     fhaNumberKeyValue=
	     	{4.6},
	     fhaCausesKeyValue=
	     	{Helicopter has to release rope}, 
	     fhaDescriptionKeyValue=
	     	{If the leash has been cut, releasing the rope without warning would be fatal.}, 
	     fhaMitigationKeyValue=
	     	{\tabitem Leash shorter than arm length\\
	     	\tabitem \textcolor{purple}{patient}/\textcolor{teal}{rescuer} can grab line with arm.\\
	     	\tabitem \textcolor{purple}{patient}/\textcolor{teal}{rescuer} has self-belay loop ready for emergency\\
	     	\tabitem \textcolor{purple}{patient}/\textcolor{teal}{rescuer} has a stable communication connection to the helicopter\\
	     	\tabitem There is enough time to attach a safe connection to the highline},
	     fhaCritPostKeyValue=
	     	{\textcolor{red}{high}. Depending on the warning time, this still represents a critical moment. 
			Whether a fuse can be attached depends on the warning time available.}]
	     
%4.7
\fhaTable[fhaSystemKeyValue=
			{Highline},
		 fhaFunctionKeyValue=
		 	{positioning},
	     fhaCritPreKeyValue=
	     	{\textcolor{red}{high}},
	     fhaNumberKeyValue=
	     	{4.7},
	     fhaCausesKeyValue=
	     	{Aerodynamic interaction of highline with helicopter downwash (swinging of the line in the wind)}, 
	     fhaDescriptionKeyValue=
	     	{If the highline swings due to the downwash when unloaded, the \textcolor{teal}{rescuer} cannot manually dampen it because he needs his hands to cut the leash.
	      	This swing can interfere with cutting or make it very difficult.}, 
	     fhaMitigationKeyValue=
	     	{\tabitem \textcolor{purple}{patient} dampens the highline manually if necessary\\
	     	\tabitem Helicopter can create a slight lateral tension in the highline to keep \textcolor{teal}{rescuer} and \textcolor{purple}{patient} out of the range of movement of the swinging highline. \\
	     	\tabitem Short leash to ensure that the highline can be reached by the \textcolor{teal}{rescuer}\\
	     	\tabitem Sheathing of line and backup to prevent loops},
	     fhaCritPostKeyValue=
	     	{\textcolor{orange}{medium}, as loop formation is prevented. \textcolor{purple}{patient} can manually dampen line.
 			It is also possible to change the position of the helicopter sideways to the line until the leash is in slight tension.}]

%4.8
\fhaTable[fhaSystemKeyValue=
			{Highline},
		 fhaFunctionKeyValue=
		 	{load transfer},
	     fhaCritPreKeyValue=
	     	{\textcolor{green}{low}},
	     fhaNumberKeyValue=
	     	{4.8},
	     fhaCausesKeyValue=
	     	{Rockfall or other objects blown down by downwash}, 
	     fhaDescriptionKeyValue=
	     	{A falling stone, branch, carabiner, etc., is enough to cut the line under tension. In this phase, the highline is no longer the load-bearing system.
	      	This means that it is less susceptible to breaking through. The breaking of the highline, which is no longer load-bearing, is not considered to have any influence.}, 
	     fhaMitigationKeyValue=
	     	{\tabitem Cleaning the anchors of loose stones and objects that could be thrown up\\
	     	\tabitem Prohibit actions near the highline that involve throwing/kicking off objects/etc.},
	     fhaCritPostKeyValue=
	     	{\textcolor{green}{low}. Very unlikely if the anchors are cleaned sufficiently and the staff at the anchors behave accordingly.
			In addition, the load transfer to the helicopter longline is completed in this phase.}]

%4.9
\fhaTable[fhaSystemKeyValue=
			{Leash},
		 fhaFunctionKeyValue=
		 	{load transfer},
	     fhaCritPreKeyValue=
	     	{\textcolor{red}{high}},
	     fhaNumberKeyValue=
	     	{4.9},
	     fhaCausesKeyValue=
	     	{Anything that can cause delays when cutting the leash, e.g., the scissors falling down, poor cutting performance, adverse conditions when cutting, etc.}, 
	     fhaDescriptionKeyValue=
	     	{In this phase, the “load transfer” function of the leash is deliberately destroyed by cutting}, 
	     fhaMitigationKeyValue=
	     	{\tabitem Only start the cutting process when the conditions allow for a quick cut\\
	     	\tabitem Only use a simple leash that can be cut easily\\
	     	\tabitem Second pair of scissors ready for \textcolor{purple}{patient}. Good communication in case the cut goes wrong\\
	     	\tabitem If problems persist, attaching of a sling to between the patient's harness and the leash ring \\},
	     fhaCritPostKeyValue=
	     	{\textcolor{red}{high}. With two people ready to cut + good communication, -> a good opportunity to keep this process safe and short.}]

The failure analysis of the harness in this phase is not discussed separately since no exceptional
load case is known.

\clearpage
\subsection{Phase 5: Departure}
\label{sec:fha:phase5}

After completing Phase 4, the load-bearing path is limited to the heli, rope, and harnesses alone. In general, the operations in this phase are considered equivalent to a climber rescue.\\
For the sake of completeness, however, it should be mentioned again that backup loops, if present, can be inadvertently wrapped around the extremities or equipment of the \textcolor{teal}{rescuer} and \textcolor{purple}{patient}. This is a critical case due to the potentially tricky dissolution and the resulting consequences. In this analysis, however, a sheathing of the line and the backups is planned. This means that this difficult-to-control danger can be eliminated from the actual scenario for an exercise.

\section{Modified highline setup for training}
\label{sec:modified}

The usual structure of a highline already results in a very high safety standard. As described in Chapter 3, each component's simple redundancy is sufficient for normal use. Nevertheless, we propose a special structure for the planned exercise scenario that deviates from the usual structure by adding additional safety components to exclude or minimize the safety concerns already discussed as best as possible.\\
For the special protection of the highline and the backup, these are threaded through a 9m long fire hose, positioned in the middle of the line in the area of the \textcolor{purple}{patient}. In addition to solid abrasion protection in the event of contact between the longline/\textcolor{teal}{rescuer} and the line, the hose also protects against falling parts, e.g., when the longline has to be released.
Furthermore, the non-tensioned, redundant backup ropes' loops are covered, preventing unintentional entanglement in these
loops.\\

To cope with the potentially greater forces acting on the entire system than in normal cases (2 people on the line; in the worst case, dynamic fall of two people including the longline into the highline; lateral deflection by helicopter), the anchor points are set up with sufficient safety reserves. The setup uses four points (12mm bolts or healthy trees (trunk diameter approx. 30cm)) and a compensating anchor. The mainline uses certified high-strength webbing (e.g., RedTube, Slacktivity, minimal breaking strength 36.9kN). For double redundancy in the event of a mainline failure, the backup system uses two static core-sheath ropes. In addition to sufficient strength, these offer a higher level of safety in the event of cutting loads from stones at the anchor point or sharp-edged material on \textcolor{teal}{rescuer}/\textcolor{purple}{patient}. \\

Instead of just one leash ring as a connection between the leash and the highline, two tested leash rings ensure redundancy in the improbable event of a material failure of one ring.

A strand of new climbing rope instead of a regular leash (climbing rope
threaded into a tubular webbing) attaches the \textcolor{purple}{patient} to the leashring, allowing easy and reliable cutting during the training. The tubular webbing around the rope makes cutting a conventional leash much more difficult. An improper cut would create the risk of the leash failing to provide a safe attachment to the highline. This situation is critical because the helicopter is tied up, but no safe connection between the \textcolor{teal}{rescuer}/\textcolor{purple}{patient} and the highline is established. Release of the longline in this situation poses a danger to personnel.

\section{Resulting procedure for training}
\label{sec:result}

Based on the error analysis, some positions in the procedure that should be expanded to increase
the safety of those involved during the exercise can be identified. The assumed procedure from Chapter 2 is repeated below.
Further coordination/elaboration with pilots and air \textcolor{teal}{rescuer}s, and further adjustments are explicitly
desired.

\begin{landscape}

\begin{longtable}{|p{4cm}|p{4cm}|p{4cm}|p{4cm}|p{4cm}|}
 \hline
 rescue phase & special features in exercise & \textcolor{teal}{rescuer's} tasks & \textcolor{purple}{patient's} tasks & \textcolor{blue}{observer's} tasks \\ 
 \hline
 \hline 
 0. Preparation & 
 \begin{itemize}
    \item Test radio connection
    \item Material ready and checked
    \item Check if all functional roles are ready
  \end{itemize} & 
   \begin{itemize}
    \item Test radio
    \item Check scissors available and accessible
  \end{itemize} & 
   \begin{itemize}
    \item Test radio
    \item Highline reachable by hand while sitting in harness
    \item Self-belay loop available and long enough
    \item Check scissors available and accessible
  \end{itemize} & 
   \begin{itemize}
    \item Test radio
    \item Unobstructed view on rescue operations
    \item countercheck \textcolor{purple}{patient}
  \end{itemize} \\
  \hline
  
 1. Approach and positioning of the helicopter &
 \begin{itemize}
    \item \textcolor{purple}{patient} and \textcolor{blue}{observer} signal abort if wrong approach or error
  \end{itemize} & 
   \begin{itemize}
    \item Abseilcheck if variable longline
    \item Guide helicopter
    \item Grab patient with legs
    \item Keep longline and highline separated
  \end{itemize} & 
   \begin{itemize}
    \item Check approach
    \item Signal rescuer with Y sign
  \end{itemize} & 
   \begin{itemize}
    \item Check approach
    \item Check positioning of helicopter
  \end{itemize} \\
  \hline
  
 2. Attaching the patient to the longline &
 \begin{itemize}
    \item Leash shorter than usual
    \item No backup loops fluttering in downwash
  \end{itemize} & 
   \begin{itemize}
    \item Safe attachment of patient's harness to the longline
    \item Reposition helicopter if necessary
    \item Hold patient with legs
    \item Keep longline and highline separated
  \end{itemize} & 
   \begin{itemize}
    \item Check helicopter positioning
    \item Check correct attachment of harness to the longline and signal to proceed
    \item Check for deviations from the procedure and interfere if observed
  \end{itemize} & 
   \begin{itemize}
    \item Monitor helicopter and rescuer position
    \item Assess general time and quality of the procedure and abort if doubts of success
  \end{itemize} \\
  \hline
  
 3. Lifting by helicopter to transfer load from highline to longline &
 \begin{itemize}
    \item Leash shorter than usual
    \item No backup loops fluttering in downwash
  \end{itemize} & 
   \begin{itemize}
    \item Instruct helicopter to produce a horizontal distance to the highline
    \item Instruct helicopter to lift for the load transfer
    \item Instruct helicopter to hover at the end of load transfer
    \item Check for any connections to the highline apart from the leash
    \item Dampen oscillations of the highline manually if they occur
  \end{itemize} & 
   \begin{itemize}
    \item Check for any connections to the highline apart from the leash
    \item Check rescuer's instructions to the helicopter
    \item Check for deviations from the procedure and interfere if observed
  \end{itemize} & 
   \begin{itemize}
    \item Monitor helicopter and rescuer position
    \item Check for any connections to the highline apart from the leash
    \item Assess general time and quality of the procedure and abort if doubts of success
  \end{itemize} \\
  \hline

 4. Cutting the leash &
 \begin{itemize}
    \item Leash shorter than usual
    \item Leash easier to cut than usual
    \item Self-belay loop to reestablish a safe connection if necessary
    \item Second pair of scissors in case of a failure
    \item No backup loops fluttering in downwash
  \end{itemize} & 
   \begin{itemize}
    \item Instruct helicopter to hover at the end of load transfer
    \item Check for any connections to the highline apart from the leash
    \item Dampen oscillations of the highline manually if they occur
    \item Cut leas if safely possible
  \end{itemize} & 
   \begin{itemize}
    \item Check for any connections to the highline apart from the leash
    \item Check rescuer's instructions to the helicopter
    \item Check if highline in neutral position before cut
    \item Hold Self-belay ready
    \item Offer own scissors if necessary
    \item Check for deviations from the procedure and interfere if observed
  \end{itemize} & 
   \begin{itemize}
    \item Monitor helicopter and rescuer position
    \item Check if highline in neutral position before cut
    \item Check for any connections to the highline apart from the leash
    \item Assess general time and quality of the procedure and abort if doubts of success
  \end{itemize} \\
  \hline

 5. Departure &
 \begin{itemize}
    \item None
  \end{itemize} & 
   \begin{itemize}
    \item Instruct helicopter to a safe departure
  \end{itemize} & 
   \begin{itemize}
    \item Check rescuer's instructions to the helicopter
  \end{itemize} & 
   \begin{itemize}
    \item Monitor helicopter and rescuer position
    \item Check rescuer's instructions to the helicopter
  \end{itemize} \\
  \hline
   
\end{longtable}
\label{tab:exercise}

\end{landscape}



\end{document}
